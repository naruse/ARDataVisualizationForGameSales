\documentclass{article}

\usepackage[hscale=0.7]{geometry}

\usepackage{lipsum}

%\pagenumbering{gobble}


\begin{document}

\title{ARMeet: AR Data vizualization for meetings.}

\author{Juan Sebastian Mu\~noz Arango \\ T00601208}
\maketitle %Tell LaTeX to print the title, author name, etc. here.

\section{Description:}
ARMeet is a mobile app used to easily visualize information contained in
reports shared in meetings.\newline

It's critical for everyone in a meeting to understand what a report
contains. And the faster this task is achieved the better as users are able
to make questions about the data contained in the report instead taking the
time to understand the data itself in the meeting.\newline

With the help of Augmented Reality, ARMeet aims to ease the understandment of
the data contained in the report and to some extent comparte data in it with
visualization techniques taking the report pages itself as image targets as
anchor points to draw the data.\newline

This project is aimed to be a proof of concept for ARMeet; the idea is to
visualize the data contained in the final document report. The data planned
to be used comes from a dataset called video games sales and ratings.

\section{Project Concept:}
\subsection{Environment:}
ARMeet is intended to be used indoors, most likely in conference
rooms. Its main purpose is to enhance the understanding of reports in
meetings to further the discussions on the data contained in the reports.

\subsection{User interaction with the application:}
ARMeet interaction with the augmented objects is limited and its purpose
is more for the data interpretation during meetings. \newline

The way a user interacts with the application is through his personal
phone pointing towards each of the report's pages and seeing the data being
augmented. With some extra image targets the user can generate comparsions
between targets.

\subsection{Step by step usage:}
Using ARMeet should be intuitive and easy to use, being said that here are
the steps to use the application:
\begin{enumerate}
\item User opens the ARMeet app and points the camera towards any of the
  report pages and data gets augmented from the report pages.
\item With the center of the screen, the user can point torwards parts of the data to
  get some relevant information from the report.
\item If the user decides to put an image target next to the augmented report
  page, then the visualized data reacts and a comparsion is done with the
  image target that just appeared in the user's camera.
\item User then can compare and make conclusions on his own with the
  generated data.
\item Finally, the user can at any point move to another page and start the
  process all over again.
\end{enumerate}

\subsection{Resources to be used:}
For this project I plan to use Vuforia for the Augmented Reality part with
Unity3D for the rendering side and the dataset from kaggle called ``Video
game sales with ratings'' from: \\\textit{https://www.kaggle.com/rush4ratio/video-game-sales-with-ratings/}.
\subsection{Target platforms:}
I plan to target this app as a start for the \textit{iOS} platform, and
if there is time for the \textit{Android} OS.
\section{Risks:}
There can be some issues when working on this project, next I will enumerate
the most critical ones that I think will hammer the completition of the project:
\begin{description}
  \item[Multiple image targets visualized at the same time.] \hfill \\
    I'm not sure how well multiple image targets work on my
    cellphone; at most I will be visualizing 3 image targets at the same time
    (the report page and 2 comparsion image targets).
  \item[Amount of image targets.] \hfill \\
    I noticed that there are different platforms for each console; The risk
    here is having to use too many image targets; this will defeat the
    purpose to have a single report to see the data.
  \item[Size of the data to visualize] \hfill \\
    Right now the data to visualize is stored in a csv file of 500kb, but I'm
    not 100\% sure how Unity can handle thousands of objects being rendered
    at the same time in a phone.

\end{description}

\section{Deliverables:}
This project output will be an application and a report (that will be used
with the application to augment the data in it).

\end{document}
