\documentclass{article}

\usepackage[hscale=0.7]{geometry}

\usepackage{lipsum}

\pagenumbering{gobble}


\begin{document}

\title{Potel RVR: Realtime Virtual Reality Pottery Shaper.}

\author{Juan Sebastian Mu\~noz Arango \\ T00601208}
\maketitle %Tell LaTeX to print the title, author name, etc. here.

\section{Description:}

\section{Project Concept:}
\subsection{Environment:}
Potel's environment is located in a small town house with a beautiful scenery where
you can sit, relax and let your imagination fly.
\subsection{User interaction with the world:}
The way the user interacts with the world is with his hands, the user can
model the clay to his will to create pots and also interact with a lever to
set the speed of the rotating pot in order to have more control on the model
being shaped. Finally the user will be able to paint the pot after it has
been ``baked''; All of this through a head mounted display to get a fully
immersive experience.
\subsection{Step by step usage:}
Modeling clay in Potel should be intuitive and fun, being said that, here are
the steps to use the application:
\begin{enumerate}
\item User sits in a chair with a leap motion on front and a Head Mounted
  Display put on.
\item User clicks on create new model and a flat cylinder is
  instantiated as an initial clay to shape.
\item Clay starts to spin and user starts to model the clay.
\item When the user is happy with what he has modeled, the user clicks on
  bake (on a menu) to move to the painting process.
\item Same happens with painting; pot starts to spin and user can then start
  to paint.
\item Finally, the user can at any point restart the whole process.
\end{enumerate}
\subsection{Resources to be used:}
For this project I plan to use both an oculus rift as a Head Mounted Display
and a leap motion to track hands for shaping the pot.
\subsection{Target platforms:}
I plan to target this application for \textit{Windows}.
\section{Project timeline:}
I have decided to have milestones each 2 weeks, with this I will have the
flexibility to work on larger deliverables over per week basis milestone and
in the case I'm stuck I can always use the weekends to catch up and meet the
proposed milestones.
\subsection{Milestones:}
\begin{description}
  \item[Monday September 21 - 2015] \hfill \\
    Oculus integration and leap integration.
  \item[Monday October 5 - 2015] \hfill \\
    Deformable meshes interacting with the mouse
  \item[Monday October 19 - 2015] \hfill \\
    Deformable meshes interacting with Leap Motion, initial menus and start working on the scenery of the project.
  \item[Monday November 2 - 2015] \hfill \\
    Final touches to the project, make the scene look good, bug fixes.
  \item[Monday November 9 - 2015] \hfill \\
    Deliver the project.
\end{description}
\subsection{Status reports:}
There are going to be 2 deliverables as follows:
\begin{enumerate}
  \item October 10: Prototype that has deformable meshes working inside the
    oculus sdk with mouse interaction, initial scenery and menus.
  \item November 9: Final project finished with leap motion integrated, full
    scene working and vertex painting implemented in the deformed mesh.
\end{enumerate}
\subsection{Completition date:}
I plan to have this completed by Monday November 9 - 2015.

\end{document}
