\section{PC}
The PC console is a special one as the dataset contains only labels for the
PC platform but it doesnt specify the OS that was designed for.\\
PC Home computer games became popular following the video game crash of 1983 leading
to the era of the "bedroom coder". In the 1990s, PC games lost mass-market
traction to console games before enjoying a resurgence in the mid-2000s
through digital distribution\cite{PC}.

In general the PC ``console'' is the one that covers most of the years; from
1985 to 2016 according to the dataset. Its not fair to have a representative
game across all these titles but its worth mentioning that the \textit{Sims
  3} published by Electronic Arts in 2009 is the game that has the most sales
with \$8.01 million dollars.\\
It is also important to note that the whole PC platform contains 974
registered games with 260.299 million dollars in sales. Curiosly enough the
PC platform doesnt have Japan sales like the other platforms. In order to see with
more detail the PC platform, point \textit{ARMeet} to Figure \ref{fig:PCImage}.

\begin{figure}[h!]
  \centering
  \centerline{\includegraphics[scale=0.25]{images/PCMainTarget.png}}
  \caption{Representative picture of the PC console, point ARMeet to this image to
    see the different sales across the years.}
  \label{fig:PCImage}
\end{figure}

%% Platform: PC
%% Games: 974
%% Total Sales: 260.299, by country US: 94.53003 EU: 142.4393 JP: 0.17 Other: 22.38012
