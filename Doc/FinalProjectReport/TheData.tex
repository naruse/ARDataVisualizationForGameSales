\section{The Data}
The data that is used in this project comes from the Kaggle webpage,
specifically the \textit{Video game sales and ratings}
dataset\cite{KaggleDataset}. This data comes in a csv file with the following
fields:\\
\texttt{Name, Platform, Year\_of\_Release, Genre, Publisher, NA\_Sales, \\
  EU\_Sales, JP\_Sales, Other\_Sales, Global\_sales, Critic\_score,\\
  Critic\_count, User\_score, User\_count, Developer, Rating}\\

From these fields, \texttt{Critic\_score, Critic\_count, User\_score} and
\texttt{User\_count} where discarded as only ~6900 out of 16720 games are
complete.

From the data processed, 6 Major consoles where extracted and a total of 31
platforms from all the consoles where found.

\subsection{The games}
As mentioned, there are a total of 16720 games; these games where parsed and
separated by their respective consoles and console platforms.\\

Each game belongs to a major console and a specific platform whithin the
console, the games are organized as a 3D column with different colors;
on the bottom of the column there is a blue color which represents the sales
if US; then there is a yellow color wich represents the sales in europe, next
is a red color which represents the sales in Japan and finally a white color
that represents the sales in other countries.\\

If one of these colors lacks in the 3D image; it means that there are no
registered sales in that country. Finally, along all the colors there is a
purple cylinder that covers all the sales across all the countries; this
color represents the total sales of that game.\\

\subsection{Consoles}
As mentioned, 6 Major consoles where extracted from the data; each console
contains a set of platforms where every game belongs to.
The extracted consoles are: Nintendo, Sony, Microsoft, SEGA, PC and Other
consoles.\\

The \textit{\textbf{Other}} console, groups all the smaller platforms that
dont have a renowed vendor.\\

Finally each of these consoles are discussed with more detail in later
sections of this report with all the platforms each of them has along with
their most significant game sales.


\subsection{The platforms}
There are a total of 31 platforms; each platform serves as a way to group
games and also helps minimizing the amount of data that needs to be shown
at the same time.\\

Some platforms contain more games than others and every platform belongs to a
console. The platforms to be visualized are organized as follows:\\
\textbf{1. Nintendo:} Nintendo Enterainment System, Super Nintendo, Gameboy,
Gameboy Advance, Nintendo DS, Nintendo 3DS, Nintendo 64, Game Cube, Wii and
Wii-U.\\
\textbf{2. Sony:} Playstation, Playstation 2, Playstation 3, Playstation 4,
Playstation Portable (PSP) and Playstation Vita.\\
\textbf{3. Microsoft:} Xbox, Xbox 360 and Xbox One.\\
\textbf{4. SEGA:} Sega Saturn, Sega CD, Sega Genesis, Sega Gamegear and Sega Dreamcast.\\
\textbf{5. PC:} PC.\\
\textbf{6. Other:} Atari 2600, Wonder Swan, Neo Geo, Turbo Grafx 16,
Panasonic 3DO and NEC PC-FX.

%% on Figure~\ref{fig:OrganizationChart} Left.

%% \begin{figure}[h]
%%   \centering
%%   \includegraphics[scale=0.5]{images/MetaOrganizationalChart.png}
%%   \caption{Left: Current organization chart, Right: Proposed new organization chart}
%%   \label{fig:OrganizationChart}
%% \end{figure}
